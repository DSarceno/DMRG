% AUTHOR: Diego Sarceno
% Last Update: 11.07.2020

\documentclass[11pt, spanish, letterpage]{article} %tipo de documento

\usepackage[letterpaper]{geometry} %margenes
\geometry{verbose,tmargin=2.5cm,bmargin=2.5cm,lmargin=2cm,rmargin=2cm}
\usepackage{amsmath,amsthm,amssymb} %modos matemáticos y  simbolos
\usepackage{latexsym,amsfonts} %simbolos matematicos
\usepackage{cancel} %hacer la linea que cancela las ecuaciones
\usepackage[spanish, es-noshorthands]{babel} %comandos en español y cambia el cuadro por la tabla
\decimalpoint %cambia las comas por puntos decimal
\usepackage[utf8]{inputenc} %caracteristicas del español
\usepackage{physics} %Simbolos fisicos
\usepackage{array} %mejores formatos de tabla
\parindent =0cm %sangria
\usepackage{graphicx} %graficas e imagenes
\usepackage{mathtools}
\usepackage[framemethod=TikZ]{mdframed}%Entornos talegas
\usepackage[colorlinks = true,
			linkcolor = blue,
			citecolor = black,
			urlcolor = blue]{hyperref}%formato de los links y URL's
\usepackage{multicol} %varias columnas
\usepackage{enumerate} %enumeraciones
\usepackage{pgf,tikz,pgfplots} %documentos en formato tikz
\usepackage{mathrsfs} %letras chingonas (transformada de laplace)
\usepackage{subfigure} %varias figuras seguidas
\usepackage[square,numbers]{natbib} %bibliografias
\usepackage[nottoc]{tocbibind}
\bibliographystyle{plainnat}
\usetikzlibrary{arrows, babel, calc}
\usepackage{tabulary}
\usepackage{multirow} %ocupar varias filas en una tabla
\usepackage{fancybox} %recuadros talegas
\usepackage{float} %ubicar graficas
\usepackage{color}
\usepackage{comment}
\usepackage{stackrel}
\usepackage{calligra}
\usepackage{lipsum} % texto de relleno
\usepackage{cite}
\usepackage{circuitikz} % crear circuitos
\usepackage{listings} % permite el ingreso de codigo
\usepackage{longtable}
%\usepackage{showframe}
%\usepackage{LobsterTwo}
% NEW PACKAGES
\usepackage{makeidx}
\usepackage{authblk} % para la manipulación de autores y afiliación
\usepackage{booktabs}
\usepackage{colortbl}
\usepackage{bbold}
\usepackage{dsfont}
\usepackage{tensor}
\usepackage{colortbl}
\usepackage{amsbsy}
\usepackage[draft,inline,nomargin]{fixme} \fxsetup{theme=color}

%This defines my comments
\definecolor{mycolor}{RGB}{0,0,250}
\FXRegisterAuthor{ds}{sds}{\color{mycolor}DS}


\usepackage{pdfpages}
\setlength{\parindent}{1cm} %sangria

%%%%%%%%%%%%%%%%%%%%%%%%%%%%%%%%%%%%%%%%%%%%%%%%%%%%%%%%%%%
\lstset{basicstyle=\ttfamily,breaklines=true}
\lstset{numbers=left, numberstyle=\tiny, stepnumber=1, numbersep=6pt}
\lstset{emph={import,as,return,for,in,else,if,def,True,False,append}, emphstyle=\color{blue}, emph={[2]pKronecker},
emphstyle={[2]\color{violet}}, emph={[3]float,input,int,range,print,len},
emphstyle={[3]\color{violet}}}
\lstset{morecomment=[l][\color{gray!40}]{\#}, morestring=[b][\color{green!50!black}]"}
%%	Importe de archivo: \lstinputlisting[inputencoding=latin1]{'nombre del archivo'.py}
%%%%%%%%%%%%%%%%%%%%%%%%%%%%%%%%%%%%%%%%%%%%%%%%%%%%%%%%%%%
\setlength{\columnseprule}{0pt}
%-------------------------------------------------
\newcommand{\N}{\mathbb{N}}
\newcommand{\Z}{\mathbb{Z}}
\newcommand{\Q}{\mathbb{Q}}
\newcommand{\I}{\mathbb{I}}
\newcommand{\R}{\mathbb{R}}
\newcommand{\C}{\mathbb{C}} %Conjuntos numericos
\newcommand{\F}{\mathbb{F}} %Campo Cualquiera
\newcommand{\Pos}{\mathbb{P}} %Reales positivos
\newcommand{\Hilbert}{\mathcal{H}} % Espacio de Hilbert
\newcommand{\f}{\textit{f}} %f de funcion
\newcommand{\g}{\textit{g}}
\newcommand{\kernel}{\mathscr{N}} %kernel
\newcommand{\range}{\mathcal{R}} %range
\newcommand{\lagran}{\mathcal{L}} %lagrangiano
\newcommand{\laplace}{\mathscr{L}} %transformada de laplace, mapas lineales
\newcommand{\partition}{\mathfrak{z}} % función de partición
\newcommand{\M}{\mathcal{M}} %Matrices
\newcolumntype{E}{>{$}c<{$}} %entorno matematico en columnas de una tabla
\newcommand{\vi}{\boldsymbol{\hat{\imath}}}
\newcommand{\vj}{\boldsymbol{\hat{\jmath}}}
\newcommand{\vk}{\vu{k}}%vectores unitarios R3
\newcommand{\vr}{\hat{r}}
\newcommand{\vp}{\boldsymbol{\hat{\phi}}}
\newcommand{\vz}{\vu{z}}%vectores unitarios en cilindricas
\newcommand{\vaz}{\boldsymbol{\hat{\theta}}}%vectores unitarios en esféricas
\newcommand{\vx}{\vu{x}}%vectores
\newcommand{\vy}{\vu{y}}%vectores 
\newcommand\numberthis{\addtocounter{equation}{1}\tag{\theequation}}
\newcommand{\LI}{\lim _{h\longrightarrow 0}}
\newcommand{\SU}{\longrightarrow \sum _{n=0} ^{\infty}}
\newcommand{\QED}{\hfill {\qed}}
\newcommand{\cis}{\text{cis} \,}
% matrices de pauli
\newcommand{\pauli}[1]{\sigma _{#1}}
%----------------------------------------------------------
%----------------------------------------------------------


%-paquete para unidades en el sistema internacional
\usepackage[load=prefix, load=abbr, load=physical]{siunitx}
\newunit{\gram}{g }%gramos
\newunit{\velocity}{ \metre / \Sec }%unidades de velocidad sistema internacional
\newunit{\acceleration}{ \metre / \Sec^2 }%unidades de aceleracion sistema internacional
\newunit{\entropy}{ \joule / \kelvin }%unidades de entropia sistema internacional
%--definiendo constantes fisicas en el SI
\newcommand{\accgravity}{9.8 \metre / \Sec^2}
%---diferencial inexacta
\newcommand{\dbar}{\mathchar'26\mkern-12mu d}
%-------------------------END-------------------------------------
%------------------------Barra negra-------------------------------
\tikzset{
	warningsymbol/.style={
		rectangle,draw=black,
		fill=white,scale=1,
		overlay}}
\mdfdefinestyle{warning}{%
	hidealllines=true,leftline=true,
	skipabove=12,skipbelow=12pt,
	innertopmargin=0.4em,%
	innerbottommargin=0.4em,%
	innerrightmargin=0.7em,%
	rightmargin=0.7em,%
	innerleftmargin=1.7em,%
	leftmargin=0.7em,%
	middlelinewidth=.2em,%
	linecolor=black,%
	fontcolor=black,%
	firstextra={\path let \p1=(P), \p2=(O) in ($(\x2,0)+0.5*(0,\y1)$)
										node[warningsymbol] {$\mathcal{S}$};},%
	secondextra={\path let \p1=(P), \p2=(O) in ($(\x2,0)+0.5*(0,\y1)$)
										node[warningsymbol] {$\mathcal{S}$};},%
	middleextra={\path let \p1=(P), \p2=(O) in ($(\x2,0)+0.5*(0,\y1)$)
										node[warningsymbol] {$\mathcal{S}$};},%
	singleextra={\path let \p1=(P), \p2=(O) in ($(\x2,0)+0.5*(0,\y1)$)
										node[warningsymbol] {$\mathcal{S}$};},%
}
%%%%%%%%%%%%%%%%%%%%%%%%%%%%%%%%%%% Tema - BEGIN
\newtheoremstyle{Tema}% name of the style to be used
  {0mm}% measure of space to leave above the theorem. E.g.: 3pt
  {10mm}% measure of space to leave below the theorem. E.g.: 3pt
  {}% name of font to use in the body of the theorem
  {}% measure of space to indent
  {\bfseries}% name of head font
  {\newline}% punctuation between head and body
  {30mm}% space after theorem head
  {}% Manually specify head

\theoremstyle{Tema} \newtheorem{Tema}{Tema} %%%%% Template para Temas
\theoremstyle{Tema} \newtheorem{serie}{Serie}              %%%%%  Template para Series de ejercicios
\theoremstyle{Tema} \newtheorem{teorema}{Teorema}              %%%%%  Template para Teoremas
\theoremstyle{Tema} \newtheorem{pregunta}{Pregunta}              %%%%%  Template para Series de ejercicios
\theoremstyle{Tema} \newtheorem{ejercicio}{Ejercicio}    %%%%%  Template para Ejercicios
\theoremstyle{Tema} \newtheorem{ejemplo}{Ejemplo}    %%%%%  Template para Ejemplos
\theoremstyle{Tema} \newtheorem{solucion}{Solución}    %%%%%  Template para Soluciones
\theoremstyle{Tema} \newtheorem{problem}{Problema}    %%%%%  Template para Problema
\theoremstyle{Tema} \newtheorem{definicion}{Definición}    %%%%%  Template para Soluciones
\theoremstyle{Tema} \newtheorem{proposicion}{Proposición}    %%%%%  Template para Soluciones
\theoremstyle{Tema} \newtheorem{lema}{Lema}    %%%%%  Template para Soluciones
%-------------------------END-------------------------------------
